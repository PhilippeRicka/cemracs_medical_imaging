% ----------------------------------------------------------------
% AMS-LaTeX Paper ************************************************
% **** -----------------------------------------------------------
\documentclass{amsart}
\usepackage{graphicx}
% ----------------------------------------------------------------
\vfuzz2pt % Don't report over-full v-boxes if over-edge is small
\hfuzz2pt % Don't report over-full h-boxes if over-edge is small
% THEOREMS -------------------------------------------------------
\newtheorem{thm}{Theorem}[section]
\newtheorem{cor}[thm]{Corollary}
\newtheorem{lem}[thm]{Lemma}
\newtheorem{prop}[thm]{Proposition}
\theoremstyle{definition}
\newtheorem{defn}[thm]{Definition}
\theoremstyle{remark}
\newtheorem{rem}[thm]{Remark}
\numberwithin{equation}{section}
% MATH -----------------------------------------------------------
\newcommand{\norm}[1]{\left\Vert#1\right\Vert}
\newcommand{\abs}[1]{\left\vert#1\right\vert}
\newcommand{\set}[1]{\left\{#1\right\}}
\newcommand{\Real}{\mathbb R}
\newcommand{\eps}{\varepsilon}
\newcommand{\To}{\longrightarrow}
\newcommand{\BX}{\mathbf{B}(X)}
\newcommand{\A}{\mathcal{A}}
% ----------------------------------------------------------------
\begin{document}

\title{Funzioni MATLAB\copyright\ per le wavelet interpolanti}%
\author{Silvia Bertoluzza}%
%\address{}%
\email{silvia.bertoluzza@imati.cnr.it}%

\maketitle
% ----------------------------------------------------------------
\section{Framework}

Le wavelet interpolanti simmetriche hanno l'equazione di dilatazione che pu\`o essere scritta nella forma seguente:
\[ \theta(x) = \theta(2x) + \sum_{k=1}^N a_k (\theta(2x - 2k - 1) + \theta(2x + 2k + 1) ). \]
Sono costruite come autocorrelazione delle funzioni scala $\varphi$ di Daubechies. I coefficienti $a_k$ verificano la relazione:

\section{Strutture dati}

\subsection*{\tt base} \`E una struttura che include tutti i dati che servono per descrivere e manipolare una base di ondine interpolanti:

\begin{itemize}
  \item {\tt base.a}: il filtro che compare nell'equazione di dilatazione per la funzione $\theta$;
  \item {\tt base.J}: il livello di raffinamento usato per costruire i vettori dei valori della funzione theta e delle sue derivate;
  \item {\tt base.L}: $\textrm{supp}(\theta)=[-L,L]$;
  \item {\tt base.P}: grado dei polinomi riproducibili esattamente
  \item {\tt base.J0}: $J_0 = \lceil \log_2(L)\rceil$ \`e il livello minimo richiesto dalle modifiche di bordo
  \item {\tt base.THETA}: matrice le cui colonne contengono i valori nei punti diadici della funzione $\theta$ e delle sue derivate prima e seconda.
\end{itemize}

\subsection*{\tt grid} \`E una struttura dati che descrive un'insieme di punti diadici. In dimensione due abbiamo
\begin{itemize}
    \item {\tt grid.j0}: livello coarse della griglia (scalare)
    \item {\tt grid.npoints}: numero di punti
    \item {\tt grid.j}: $(x,y)=2^{-j}(k_x,k_y) \rightarrow j$
    \item {\tt grid.kx}: $(x,y)=2^{-j}(k_x,k_y) \rightarrow k_x$
    \item {\tt grid.ky}: $(x,y)=2^{-j}(k_x,k_y) \rightarrow k_y$
    \item {\tt grid.x}: $(x,y)=2^{-j}(k_x,k_y) \rightarrow x$
    \item {\tt grid.y}: $(x,y)=2^{-j}(k_x,k_y) \rightarrow y$
    \item {\tt grid.b}: $1$ if the point is on the boundary, $0$ otherwise. NB. It could be used to distinguish different boundary portions.
\end{itemize}
\begin{rem}(Nota $\times$ me) non c'\`e per ora il tipo, perch\'e ho usato solo le funzioni di tipo 2)\end{rem}
\begin{rem} a meno che $j=j_0$, la coppia $(k_x,k_y)$ non pu\`o essere formata da due numeri pari. \end{rem}
\begin{rem} Grazie alla corrispondenza tra punti e funzioni, la struttura {\tt grid} viene usata anche per rappresentare lo spazio discreto generato dalle funzioni corrispondenti ai punti.\end{rem}

\section{Funzioni}

\subsection*{\tt start} Dato {\tt nw} (indicativo della wavelet di Daubechies di partenza) ed un livello massimo {\tt J} costruisce una struttura {\tt base}. Uso:

\

{\tt base = start(nw,J)}

\

Remark: {\tt nw} $>$ ? $ \Rightarrow \theta \in C^2$,  {\tt nw} $>$ ?  $ \Rightarrow \theta \in C^2$,  {\tt nw} $>$ ? $ \Rightarrow \theta \in C^2$


\subsection*{\tt uniform2d} Costruisce una struttura di tipo {\tt grid} corrispondente ad una griglia uniforme di passo $2^{-j_{max}}$. La griglia \`e organizzata in scale multiple, con livello coarse $= 2^{-J_0}$. Uso:

\

{\tt >> grid=uniform2d(j0,jmax) }

\

\begin{rem} di solito si usa $j_0$ = {\tt base.J0}. \end{rem}


\subsection*{\tt operator2d} Costruisce la matrice di rigidita' ottenuta tramite metodo di collocazione usando lo spazio generato dalle funzioni di {\tt space} (struttura di tipo grid) e collocando nei punti della griglia {\tt grid} per un operatore di tipo
\[
Au = \sum_{i=0}^{2} \sum_{j=0}^{2-i} c_{i,j} \partial_x^i \partial_y^j u
\]
I coefficienti $c_{i,j}$ sono passati tramite una matrice $2\times 2$ {\tt coefs}: {\tt coefs(i+1,j+1)}$=c_{i,j}$
\begin{rem}
Esempi di matrici per operatori particolari:

\[
 u \leftrightarrow
\left(
  \begin{array}{ccc}
    1 & 0 & 0 \\
    0 & 0 & 0 \\
    0 & 0 & 0 \\
  \end{array}
\right)
\]
\[
-\triangle u \leftrightarrow \left(
  \begin{array}{ccc}
    0 & 0 & -1 \\
    0 & 0 & 0 \\
    -1 & 0 & 0 \\
  \end{array}
\right)
\]
\[
\beta_x u_x + \beta_y u_y \leftrightarrow \left(
  \begin{array}{ccc}
    0 & \beta_x & 0 \\
    \beta_y & 0 & 0 \\
    0 & 0 & 0 \\
  \end{array}
\right)
\]


In generale la corrispondenza di coefficienti e operatori differenziali e' questa
\[
\left(
  \begin{array}{ccc}
    Id & \partial_x & \partial_{xx} \\
    \partial_y & \partial_{xy} & \partial_{xxy} \\
    \partial_{yy} & \partial_{xyy} & \partial_{xxyy} \\
  \end{array}
\right)
\]

\begin{rem}To do: sarebbe carino poter passare dei coefficienti che sono delle funzioni per poter trattare equazioni a coefficienti non costanti.\end{rem}
\end{rem}


% ----------------------------------------------------------------
%\bibliographystyle{amsplain}
%\bibliography{}
\end{document}
% ----------------------------------------------------------------